\documentclass[a4paper,10pt]{article}

\usepackage[hidelinks]{hyperref}
\usepackage{float}

% Lingua 
\usepackage[utf8]{inputenc}
\usepackage[italian]{babel}

% Margini
\usepackage{geometry}
\geometry{a4paper, top=3cm, bottom=3cm, left=3cm, right=3cm}
\setlength{\parindent}{0pt}

\usepackage{graphicx}   % per immagini
\usepackage{hyperref}   % per link
\usepackage{caption}    % per didascalie
\captionsetup[table]{labelformat=empty}
\usepackage{ifthen}     % per \ifthenelse


\usepackage{array}      % per tabelle
\newcolumntype{M}[1]{>{\centering\arraybackslash}m{#1}}

\usepackage{datetime}

% Definisci un formato personalizzato
\newdateformat{italianformat}{\THEDAY~\monthname[\THEMONTH]~\THEYEAR}
\newdate{dataverbale}{20}{11}{2025}

% Definizione del formato con le barre
\newdateformat{slashdate}{\twodigit{\THEDAY}/\twodigit{\THEMONTH}/\THEYEAR}

% Per logo in ogni pagina
\usepackage{eso-pic}
\usepackage{tikz}
\AddToShipoutPictureBG{%
  \ifthenelse{\value{page}>1}{%
    \begin{tikzpicture}[remember picture, overlay]
      % Icona
      \node[anchor=north west, inner sep=0pt] at ([xshift=28pt,yshift=-28pt]current page.north west) 
        {\includegraphics[width=1cm, height=1cm]{resources/Tool-8_icon_white.png}};
      
      % Riga
      \draw[black, line width=0.5pt] ([xshift=28pt + 1cm, yshift=-28pt - 0.6cm]current page.north west) 
            -- ++(3.2cm, 0);
      % Testo
      \node[anchor=west, text=black, font=\small] at ([xshift=28pt + 1cm, yshift=-28pt - 0.4cm]current page.north west) 
        {Norme di progetto};
    \end{tikzpicture}
  }{}
}


%------Ridefinizione maketitle-------
\makeatletter
\renewcommand{\maketitle}{%
  \vspace*{\fill}
  \begin{center}
    {\LARGE \bfseries \@title \par}
    \vskip 1em
    {\large \@author \par}
    \vskip 1em
    {\large\slashdate{\displaydate{dataverbale}} \par}
  \end{center}
  \vspace*{\fill}
}
\makeatother
%-----------------------------------

\title{%
  \includegraphics[width=10cm]{resources/Tool-8_white_2.png} \\[1.5ex]
  \textbf{\LARGE Norme di progetto}
}
\author{}
\date{\dataverbale}
\begin{document}
\tikz[remember picture,overlay] \node[opacity=0.2,inner sep=0pt] at (current page.center){\includegraphics[width=\paperwidth,height=\paperheight]{resources/Abstract_lines.png}};

\maketitle

\newpage

\section*{Revisioni}
\begin{table}[h!]
\centering
\begin{tabular}{|M{0.8cm}|p{2.2cm}|p{5.7cm}|M{1.8cm}|p{2.2cm}|}
  \hline
  \textbf{Ver.} & \textbf{Autore} & \textbf{Modifiche} & \textbf{Data} & \textbf{Verifica} \\
  \hline
  0.2 & Stefano Maso & Stesura fino a sezione 2.4 e correzione prima stesura & 25/11/2025 &  \\
  \hline
  0.1 & Stefano Maso & Prima versione & 20/11/2025 &  \\
  \hline
\end{tabular}
\label{tab:revisioni}
\end{table}

\newpage

\tableofcontents

\newpage

\section{Introduzione}
\subsection{Scopo del documento}
Lo scopo del seguente documento è quello di definire il Way of Working e tutte le \textit{best practices} adottate da parte del gruppo, così da garantire omogeneità del lavoro.

\subsection{Scopo del prodotto}
Il prodotto finale propone di realizzare un software che permetta la scrittura di testo con marcatori in formato MarkDown, la sua visualizzazione formattata correttamente e l'interazione con un LLM. In particolare,l'utilizzo di LLM verrà sfruttato da Second Brain per fornire funzionalità di riassunto, riscrittura, traduzione e generazione di testo in aggiunta a una forma di critica secondo il modello dei "Sei Cappelli per Pensare" di Edward De Bono.  

\section{Processi primari}
\subsection{Descrizione}
Con "processi primari" si intendono "i processi che hanno come clienti soggetti che sono esterni al gruppo".

\subsection{Acquisizione}
\subsubsection{Descrizione}
Nel processo detto di acquisizione avviene la raccolta di tutti i requisiti e la loro comprensione, così da poter individuare un capitolato adatto al gruppo.
\subsubsection{Valutazione dei capitolati}
Il gruppo ha esaminato le proposte presentate dai vari proponenti sulla base della loro complessità, delle tecnologie necessarie e dell’effettivo interesse che tali progetti suscitano nei membri del gruppo.
\subsubsection{Aggiudicazione appalto}
Il gruppo si è candidato per il capitolato C6 dell'azienda Zucchetti. Dopo un primo riscontro negativo, il gruppo è riuscito ad aggiudicarsi l'appalto desiderato.

\subsection{Fornitura}
\subsubsection{Descrizione}
Il processo di fornitura consiste nel chiarire tutti i dubbi del proponente per quanto riguarda il prodotto finale atteso. Il gruppo si impegna nel comunicare con l'azienda così da determinare i requisiti da dover soddisfare e ricevere riscontri in fase di sviluppo su quanto è stato fatto. 
\subsubsection{Glossario}
Il documento Glossario ha l'obiettivo di chiarire il significato dei diversi termini utilizzati nella documentazione, in modo da evitare fraintendimenti. È destinato a membri del gruppo, committenti e proponenti.
\subsubsection{Piano di Progetto}
Il documento Piano di Progetto si tratta di uno strumento di pianificazione per le attività che devono essere svolte in modo da rispettare la data di consegna prefissata. È composto da:
\begin{itemize}
    \item Analisi dei Rischi, ovvero tutte le difficoltà che il gruppo potrebbe incontrare durante lo svolgimento del progetto
    \item Modello di sviluppo
    \item Pianificazione
    \item Preventivo
    \item Consuntivo, l'andamento del gruppo rispetto al preventivo.
\end{itemize}
\subsubsection{Piano di Qualifica}
Il documento Piano di Qualifica serve a elencare le attività svolte dal verificatore per garantire la qualità del prodotto finale. È composto da:
\begin{itemize}
    \item Qualità di processo
    \item Qualità di prodotto
    \item Test: eseguiti sul prodotto per garantire che tutti i requisiti siano soddisfatti
    \item Resoconto
\end{itemize}
\subsubsection{Rilascio}
Una volta terminato, il prodotto verrà collaudato. Se supererà il collaudo verrà consegnato al committente con tutta la documentazione del progetto.

\subsection{Sviluppo}
\subsubsection{Descrizione}
L'obiettivo del processo di sviluppo è quello di elencare le attività da svolgere in modo da raggiungere i requisiti del prodotto.
\subsubsection{Analisi dei Requisiti}
L'Analisi dei Requisiti è il documento in cui vengono elencati tutti i requisiti del prodotto finale. Serve anche come documentazione del prodotto, in quanto contiene tutte le relative funzionalità. \\
Il documento di Analisi dei Requisiti contiene: 
\begin{itemize}
  \item \textbf{Attori}, vengono definiti gli attori 
  \item \textbf{Casi d'uso}, vengono descritti i casi d'uso, ciascuno con un diagramma a esso associato e la sua descrizione che segue una struttura standard (Attori, Precondizioni, Postcondzioni, Scenario primario e Scenari alternativi)
  \item \textbf{Requisiti}, verranno infine definiti i requisiti, che possono essere funzionali, di qualità o requisiti di vincolo
\end{itemize}
\subsubsection{Progettazione}
\subsubsection{Codifica}

\section{Processi di supporto}

\end{document}