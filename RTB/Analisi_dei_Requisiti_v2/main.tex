\documentclass[a4paper,10pt]{article}

\usepackage[hidelinks]{hyperref}
\usepackage{float}

% Lingua 
\usepackage[utf8]{inputenc}
\usepackage[italian]{babel}

% Margini
\usepackage{geometry}
\geometry{a4paper, top=3cm, bottom=3cm, left=3cm, right=3cm}
\setlength{\parindent}{0pt}

\usepackage{graphicx}   % per immagini
\usepackage{hyperref}   % per link
\usepackage{caption}    % per didascalie
\captionsetup[table]{labelformat=empty}
\usepackage{ifthen}     % per \ifthenelse

\usepackage{tikz}
\usepackage{tikz-uml}

\usepackage{array}      % per tabelle
\newcolumntype{M}[1]{>{\centering\arraybackslash}m{#1}}

\usepackage{datetime}

% Definisci un formato personalizzato
\newdateformat{italianformat}{\THEDAY~\monthname[\THEMONTH]~\THEYEAR}
\newdate{dataverbale}{11}{11}{2025}

% Definizione del formato con le barre
\newdateformat{slashdate}{\twodigit{\THEDAY}/\twodigit{\THEMONTH}/\THEYEAR}

% Per logo in ogni pagina
\usepackage{eso-pic}
\usepackage{tikz}
\AddToShipoutPictureBG{%
  \ifthenelse{\value{page}>1}{%
    \begin{tikzpicture}[remember picture, overlay]
      % Icona
      \node[anchor=north west, inner sep=0pt] at ([xshift=28pt,yshift=-28pt]current page.north west) 
        {\includegraphics[width=1cm, height=1cm]{resources/Tool-8_icon_white.png}};
      
      % Riga
      \draw[black, line width=0.5pt] ([xshift=28pt + 1cm, yshift=-28pt - 0.6cm]current page.north west) 
            -- ++(3.2cm, 0);
      % Testo
      \node[anchor=west, text=black, font=\small] at ([xshift=28pt + 1cm, yshift=-28pt - 0.4cm]current page.north west) 
        {Analisi dei requisiti};
    \end{tikzpicture}
  }{}
}


%------Ridefinizione maketitle-------
\makeatletter
\renewcommand{\maketitle}{%
  \vspace*{\fill}
  \begin{center}
    {\LARGE \bfseries \@title \par}
    \vskip 1em
    {\large \@author \par}
    \vskip 1em
    {\large\slashdate{\displaydate{dataverbale}} \par}
  \end{center}
  \vspace*{\fill}
}
\makeatother
%-----------------------------------

\title{%
  \includegraphics[width=10cm]{resources/Tool-8_white_2.png} \\[1.5ex]
  \textbf{\LARGE Analisi dei requisiti}
}
\author{}
\date{\dataverbale}
\begin{document}
\tikz[remember picture,overlay] \node[opacity=0.2,inner sep=0pt] at (current page.center){\includegraphics[width=\paperwidth,height=\paperheight]{resources/Abstract_lines.png}};

\maketitle

\newpage

\section*{Revisioni}
\begin{table}[h!]
\centering
\begin{tabular}{|M{0.8cm}|p{2.2cm}|p{5.7cm}|M{1.8cm}|p{2.2cm}|}
  \hline
  \textbf{Ver.} & \textbf{Autore} & \textbf{Modifiche} & \textbf{Data} & \textbf{Verifica} \\
  \hline
  0.4 & Besnik Murtezan & Revisione casi d'uso UC1, UC2, UC3 e verifica Sezioni 1, 2, & 15/11/2025 & - \\
  \hline
  0.3 & Stefano Maso & Stesura casi d'uso (UC1, UC5, UC6, UC7, UC8) e correzione sezione 2  & 15/11/2025 & - \\
  \hline
  0.2 & Stefano Maso & Stesura primi casi d'uso (UC2, UC3, UC4)  & 13/11/2025 & Besnik Murtezan \\
  \hline
  0.1 & Stefano Maso & Prima versione, stesura della sezione 1 e 2 (Introduzione e Descrizione) e inizio della sezione 3 (Casi d'uso)  & \slashdate{\displaydate{dataverbale}} & Besnik Murtezan \\
  \hline
\end{tabular}
\label{tab:revisioni}
\end{table}

\newpage

\tableofcontents

\newpage

\section{Introduzione}
\subsection{Scopo del documento}
Lo scopo del seguente documento è quello di delineare i requisiti e i casi d'uso del progetto \textbf{Second Brain} (capitolato d'appalto C6), sulla base della presentazione del capitolato e incontri realizzati direttamente con il proponente, per chiarire i dubbi sorti.
\subsection{Scopo del prodotto}
Il prodotto finale propone di realizzare un software che permetta la scrittura di testo con marcatori in formato MarkDown, la sua visualizzazione formattata correttamente e l'interazione con un LLM. In particolare,l'utilizzo di LLM verrà sfruttato da Second Brain per fornire funzionalità di riassunto, riscrittura, traduzione e generazione di testo in aggiunta ad una forma di critica secondo il modello dei "Sei Cappelli per Pensare" di Edward De Bono.  

\section{Descrizione}
\subsection{Obiettivi del prodotto}
Il nostro prodotto punta a sviluppare un'applicazione web che vada ad estendere le possibilità di note-taking classiche, andando ad aggiungere le potenzialità dei LLM nel creare e migliorare testi. 
\subsection{Funzionalità del prodotto}
Le funzionalità offerte dal prodotto saranno:
\begin{itemize}
    \item \textbf{Editing di testo con marcatori MarkDown}: L'utente potrà scrivere in un'area che accetta testo e marcatori in formato MarkDown;
    \item \textbf{Visualizzazione testo correttamente formattato}: L'utente sarà in grado di vedere lo stile sotteso dai marcatori presenti nel testo;
    \item \textbf{Utilizzo LLM}; L'utente, tramite un LLM, potrà riassumere, riscrivere o tradurre il testo;
    \item \textbf{Critica secondo il modello dei "sei cappelli per pensare"}: L'utente, sempre tramite LLM, avrà la possibilità di ricevere critiche sul testo scritto, prendendo spunto dalla tecnica dei "sei cappelli per pensare";
    \item \textbf{Creazione intero documento tramite LLM}: L'utente potrà far sviluppare l'intero testo all'LLM stesso;
\end{itemize}

\section{Casi d'uso}
Nella seguente sezione verranno analizzati tutti i casi d'uso individuati esaminando il capitolato d'appalto e discutendo con la proponente. \\Ciascun caso d'uso sarà definito da un diagramma UML e da una descrizione testuale contenente: attore principale, attori secondari (ove presenti), precondizioni, postcondizioni, scenario principale e scenari alternativi (ove presenti).

\subsection{Attori}
Descrizione attori
    \subsubsection{Attori Principali}
        \begin{itemize}
            \item \textbf{Utente}: persona che accede alle note all'interno del sistema 
        \end{itemize}
    \subsubsection{Attori Secondari}
        \begin{itemize}
            \item \textbf{LLM}: Large Language Model esterno che verrà integrato per fornire le funzionalità AI
            \item \textbf{Server Locale}: Server locale sul quale verranno preservate ed accedute le note
        \end{itemize}

\subsection{UC1 - Scrittura testo}
\begin{center}
\begin{tikzpicture} 
\begin{umlsystem}[x=6, fill=white!10]{Sistema}
  \umlusecase[x=0, y=4, name=scrittura]{UC1 - Scrittura testo}
\end{umlsystem}

% Attori
\umlactor[x=1, y=4]{Utente}

% Associazioni Utente
\umlassoc{Utente}{scrittura}
\end{tikzpicture}
\end{center}

\textbf{Attori principali}
\begin{itemize}
    \item Utente
\end{itemize}
\textbf{Precondizioni}
\begin{itemize}
    \item L'utente è connesso al sistema
    \item E' stata aperta una nota
\end{itemize}
\textbf{Postcondizioni}
\begin{itemize}
    \item Il testo aggiunto dall'utente è presente nell'area di editing 
\end{itemize}
\textbf{Scenario principale}
\begin{enumerate}
    \item L'utente scrive del testo nell'area dedicata all'editing secondo la sintassi MarkDown
\end{enumerate}

\subsection{UC2 - Visualizzazione testo formattato}
\begin{center}
\begin{tikzpicture} 
\begin{umlsystem}[x=6, fill=white!10]{Sistema}
  \umlusecase[x=0, y=4, name=vis]{UC2 - Visualizzazione testo formattato}
\end{umlsystem}

% Attori
\umlactor[x=0, y=4]{Utente}

% Associazioni Utente
\umlassoc{Utente}{vis}
\end{tikzpicture}
\end{center}

\textbf{Attori principali}
\begin{itemize}
    \item Utente
\end{itemize}
\textbf{Precondizioni}
\begin{itemize}
    \item L'utente è connesso al sistema
    \item E' stata aperta una nota
\end{itemize}
\textbf{Postcondizioni}
\begin{itemize}
    \item Il testo presente nell'area di editing viene formattato e visualizzato dentro un'area di output adiacente
\end{itemize}
\textbf{Scenario principale}
\begin{enumerate}
    \item L'utente visualizza il testo formattato nell'area di output
\end{enumerate}

\subsection{UC3 - Riassunto testo con LLM}
\begin{center}
\begin{tikzpicture} 
\begin{umlsystem}[x=6, fill=white!10]{Sistema}
  \umlusecase[x=0, y=4, name=riassumere]{UC3 - Riassumere testo con LLM}
\end{umlsystem}

% Attori
\umlactor[x=1, y=4]{Utente}
\umlactor[x=11, y=4]{LLM}

% Associazioni Utente
\umlassoc{Utente}{riassumere}
\umlassoc{LLM}{riassumere}
\end{tikzpicture}
\end{center}

\textbf{Attori principali}
\begin{itemize}
    \item Utente non registrato
\end{itemize}
\textbf{Attori secondari}
\begin{itemize}
    \item LLM
\end{itemize}
\textbf{Precondizioni}
\begin{itemize}
    \item L'utente è connesso al sistema
    \item E' stata aperta una nota
    \item L'utente seleziona una porzione del testo o l'intero testo dentro l'area di editing
\end{itemize}
\textbf{Postcondizioni}
\begin{itemize}
    \item Il testo selezionato è stato riassunto 
\end{itemize}
\textbf{Scenario principale}
\begin{enumerate}
    \item L'utente attiva la funzionalità "Riassumi testo"
    \item Il sistema invia la richiesta all'API dell'LLM
    \item Il sistema riceve la risposta dall'LLM
    \item Il sistema mostra il riassunto suggerito dall'LLM
    \item L'utente decide se sostituire il testo precedentemente selezionato o aggiungerlo in coda
\end{enumerate}
\textbf{Estensioni}
\begin{itemize}
    \item Nel caso in cui l'utente non ritenga opportuno il riassunto suggerito:
    \begin{enumerate}
        \item L'utente rifiuta il riassunto suggerito
        \item Il sistema ritorna alle precondizioni (mantenendo quindi la selezione del testo) 
    \end{enumerate}
\end{itemize}

\subsection{UC4 - Riscrittura testo con LLM}
\begin{center}
\begin{tikzpicture} 
\begin{umlsystem}[x=6, fill=white!10]{Sistema}
  \umlusecase[x=0, y=4, name=riscrittura]{UC4 - Riscrittura testo con LLM}
\end{umlsystem}

% Attori
\umlactor[x=1, y=4]{Utente}
\umlactor[x=11, y=4]{LLM}

% Associazioni Utente
\umlassoc{Utente}{riscrittura}
\umlassoc{LLM}{riscrittura}
\end{tikzpicture}
\end{center}


\textbf{Attori principali}
\begin{itemize}
    \item Utente
\end{itemize}
\textbf{Attori secondari}
\begin{itemize}
    \item LLM
\end{itemize}
\textbf{Precondizioni}
\begin{itemize}
    \item L'utente è connesso al sistema
    \item E' stata aperta una nota
    \item L'utente seleziona una porzione del testo o l'intero testo
\end{itemize}
\textbf{Postcondizioni}
\begin{itemize}
    \item Il testo selezionato viene riscritto in modo diverso
\end{itemize}
\textbf{Scenario principale}
\begin{enumerate}
    \item L'utente attiva la funzionalità "Riscrivi Testo"
    \item L'utente seleziona almeno uno tra i seguenti aspetti per definire la riscrittura: 
    \begin{itemize}
        \item Miglioramento Lessico 
        \item Correzione Grammatica
        \item Variazione Stile Espositivo
        \item Modifica Estensione Testo 
    \end{itemize}
    \item Il sistema invia la richiesta all'API dell'LLM
    \item Il sistema riceve la risposta dall'LLM
    \item Il sistema mostra il risultato all'utente
    \item L'utente decide se sostituire il testo precedentemente selezionato o aggiungerlo in coda
\end{enumerate}
\textbf{Estensioni}
\begin{itemize}
    \item Nel caso in cui l'utente non ritenga opportuna la riscrittura proposta:
    \begin{enumerate}
        \item L'utente rifiuta il risultato proposto
        \item Il sistema ritorna alle precondizioni (mantenendo quindi la selezione del testo) 
    \end{enumerate}
\end{itemize}

\subsection{UC5 - Traduzione testo con LLM}
\begin{center}
\begin{tikzpicture} 
\begin{umlsystem}[x=6, fill=white!10]{Sistema}
  \umlusecase[x=0, y=4, name=traduzione]{UC5 - Traduzione testo con LLM}
\end{umlsystem}

% Attori
\umlactor[x=1, y=4]{Utente}
\umlactor[x=11, y=4]{LLM}

% Associazioni Utente
\umlassoc{Utente}{traduzione}
\umlassoc{LLM}{traduzione}
\end{tikzpicture}
\end{center}
\textbf{Attori principali}
\begin{itemize}
    \item Utente
\end{itemize}
\textbf{Attori secondari}
\begin{itemize}
    \item LLM
\end{itemize}
\textbf{Precondizioni}
\begin{itemize}
    \item L'utente è connesso al sistema
    \item L'utente seleziona una porzione del testo o l'intero testo
\end{itemize}
\textbf{Postcondizioni}
\begin{itemize}
    \item Il testo viene tradotto nella lingua selezionata
\end{itemize}
\textbf{Scenario principale}
\begin{enumerate}
    \item L'utente attiva la funzionalità "Traduci Testo"
    \item L'utente seleziona la lingua in cui desidera tradurre il testo selezionato
    \item Il sistema invia la richiesta all'API dell'LLM
    \item Il sistema riceve la risposta dall'LLM
    \item Il sistema mostra il risultato all'utente
    \item L'utente decide se sostituire il testo precedentemente selezionato o aggiungerlo in coda
\end{enumerate}
\textbf{Estensioni}
\begin{itemize}
    \item Nel caso in cui l'utente non ritenga opportuna la riscrittura proposta:
    \begin{enumerate}
        \item L'utente rifiuta il risultato proposto
        \item Il sistema ritorna alle precondizioni (mantenendo quindi la selezione del testo) 
    \end{enumerate}
\end{itemize}

\subsection{UC6 - Generazione Testo con LLM}
\begin{center}
\begin{tikzpicture} 
\begin{umlsystem}[x=6, fill=white!10]{Sistema}
  \umlusecase[x=0, y=4, name=creazione]{UC6 - Generazione Testo con LLM }
\end{umlsystem}

% Attori
\umlactor[x=0, y=4]{Utente}
\umlactor[x=12, y=4]{LLM}

% Associazioni Utente
\umlassoc{Utente}{traduzione}
\umlassoc{LLM}{traduzione}
\end{tikzpicture}
\end{center}
\textbf{Attori principali}
\begin{itemize}
    \item Utente
\end{itemize}
\textbf{Attori secondari}
\begin{itemize}
    \item LLM
\end{itemize}
\textbf{Precondizioni}
\begin{itemize}
    \item L'utente è connesso al sistema
    \item E' stata aperta una nota
    \item L'utente seleziona il punto in cui vuole che venga generato il testo
\end{itemize}
\textbf{Postcondizioni}
\begin{itemize}
    \item Viene inserito il testo generato sulla base della descrizione fatta dall'utente
\end{itemize}
\textbf{Scenario principale}
\begin{enumerate}
    \item L'utente attiva la funzionalità "Genera Testo"
    \item L'utente definisce un prompt contenente la descrizione del testo da generare
    \item Il sistema invia la richiesta all'API dell'LLM
    \item Il sistema riceve la risposta dall'LLM
    \item Il sistema mostra il testo generato all'utente
\end{enumerate}
\textbf{Estensioni}
\begin{itemize}
    \item Nel caso in cui l'utente non ritenga opportuno il testo generato:
    \begin{enumerate}
        \item L'utente rifiuta il risultato proposto
        \item Il sistema ritorna alle precondizioni
    \end{enumerate}
\end{itemize}

\subsection{UC7 - Salvataggio nota}
\begin{center}
\begin{tikzpicture} 
\begin{umlsystem}[x=6, fill=white!10]{Sistema}
  \umlusecase[x=0, y=4, name=salvataggio]{UC6 - Salvataggio nota}
\end{umlsystem}

% Attori
\umlactor[x=0, y=4]{Utente}
\umlactor[x=11, y=4,scale = 0.75]{Server Locale}

% Associazioni Utente
\umlassoc{Utente}{salvataggio}
\umlassoc{Server Locale}{salvataggio}
\end{tikzpicture}
\end{center}
\textbf{Attori principali}
\begin{itemize}
    \item Utente
\end{itemize}
\textbf{Attori secondari}
\begin{itemize}
    \item Server Locale
\end{itemize}
\textbf{Precondizioni}
\begin{itemize}
    \item L'utente è connesso al sistema
    \item L'utente ha una nota aperta
\end{itemize}
\textbf{Postcondizioni}
\begin{itemize}
    \item La nota viene salvata sul Server locale
\end{itemize}
\textbf{Scenario principale}
\begin{enumerate}
    \item L'utente attiva la funzionalità "Salva Nota"
    \item Il sistema salva la nota sul Server
\end{enumerate}
\textbf{Estensioni}
\begin{itemize}
    \item Nel caso in cui la nota inizialmente aperta dall'utente avesse successivamente subito delle modifiche nel Server:
    \begin{enumerate}
        \item Il sistema avvisa la presenza di modifiche nel Server non contenute nella versione corrente dell'utente
        \item Il sistema propone all'utente la scelta tra:
        \begin{itemize}
            \item Sovrascrivere la nota presente sul Server con la versione corrente dell'utente
            \item Salvare la nota con un nome differente
            \item Aggiornare la nota corrente dell'utente con la versione presente sul Server
        \end{itemize}
        \item L'utente seleziona una tra le opzioni
        \item Il sistema effettua l'operazione scelta dall'utente
    \end{enumerate}
\end{itemize}

\subsection{UC8 - Implementazione modello "sei cappelli"}
\begin{center}

\begin{tikzpicture}
    \begin{umlsystem}[x=0,y=0]{UC8 - Implementazione modello "sei cappelli"}
    \tikzumlset{font=\small}
        \umlactor[x=-7,y=-5,scale=1.5]{Utente}
        \umlactor[x=6,y=-5,scale=1]{LLM}
        
        \umlusecase[name=UC251,x=0,y=0,align=center]{\textbf{UC8.1 - Visione Emotiva} \\ (Cappello Rosso)}
        \umlusecase[name=UC252,x=0,y=-2,align=center]{\textbf{UC8.2 - Visione Neutra} \\ (Cappello Bianco)}
        \umlusecase[name=UC253,x=0,y=-4,align=center]{\textbf{UC8.3 - Visione Ottimista} \\ (Cappello Giallo)}
        \umlusecase[name=UC254,x=0,y=-6,align=center]{\textbf{UC8.4 -Visione \textit{Devil's Advocate}} \\ (Cappello Nero)}
        \umlusecase[name=UC255,x=0,y=-8,align=center]{\textbf{UC8.5 -Visione Originalità} \\ (Cappello Verde)}
        \umlusecase[name=UC256,x=0,y=-10,align=center]{\textbf{UC8.6 -Visione Logica e Strutturata} \\ (Cappello Blu)}
        
        \umlassoc{Utente}{UC251}
        \umlassoc{Utente}{UC252}
        \umlassoc{Utente}{UC253}
        \umlassoc{Utente}{UC254}
        \umlassoc{Utente}{UC255}
        \umlassoc{Utente}{UC256}
        \draw [tikzuml association style] (LLM) -- (4.1,-5);
        
    \end{umlsystem}
\end{tikzpicture}
\end{center}

\subsubsection{UC8.1 - Visione Emotiva}
\textbf{Attori principali}
\begin{itemize}
    \item Utente
\end{itemize}
\textbf{Attori secondari}
\begin{itemize}
    \item LLM
\end{itemize}
\textbf{Precondizioni}
\begin{itemize}
    \item L'utente è connesso al sistema
\end{itemize}
\textbf{Postcondizioni}
\begin{itemize}
    \item Il testo viene riscritto in modo più emozionale
\end{itemize}
\textbf{Scenario principale}
\begin{enumerate}
    \item L'utente attiva il comando che riguardo il "Cappello Rosso"
    \item La nota viene riscritta in modo più emozionale
\end{enumerate}

\subsubsection{UC8.2 - Visione Neutra}
\textbf{Attori principali}
\begin{itemize}
    \item Utente
\end{itemize}
\textbf{Attori secondari}
\begin{itemize}
    \item LLM
\end{itemize}
\textbf{Precondizioni}
\begin{itemize}
    \item L'utente è connesso al sistema
\end{itemize}
\textbf{Postcondizioni}
\begin{itemize}
    \item L'utente viene informato se il testo da lui scritto è neutro ed imparziale o meno
\end{itemize}
\textbf{Scenario principale}
\begin{enumerate}
    \item L'utente attiva il comando che riguardo il "Cappello Bianco"
    \item Il sistema mostra se il testo da lui scritto è neutro ed imparziale o meno
\end{enumerate}

\subsubsection{UC8.3 - Visione Ottimista}
\textbf{Attori principali}
\begin{itemize}
    \item Utente
\end{itemize}
\textbf{Attori secondari}
\begin{itemize}
    \item LLM
\end{itemize}
\textbf{Precondizioni}
\begin{itemize}
    \item L'utente è connesso al sistema
\end{itemize}
\textbf{Postcondizioni}
\begin{itemize}
    \item VERIFICA o RISCRIVE chiedere con MAIL
\end{itemize}
\textbf{Scenario principale}
\begin{enumerate}
    \item L'utente attiva il comando che riguardo il "Cappello Giallo"
    \item ...
\end{enumerate}

\subsubsection{UC8.4 - Visione Devil's Advocate}
\textbf{Attori principali}
\begin{itemize}
    \item Utente
\end{itemize}
\textbf{Attori secondari}
\begin{itemize}
    \item LLM
\end{itemize}
\textbf{Precondizioni}
\begin{itemize}
    \item L'utente è connesso al sistema
\end{itemize}
\textbf{Postcondizioni}
\begin{itemize}
    \item VERIFICA o RISCRIVE chiedere con MAIL
\end{itemize}
\textbf{Scenario principale}
\begin{enumerate}
    \item L'utente attiva il comando che riguardo il "Cappello Nero"
    \item ...
\end{enumerate}

\subsubsection{UC8.5 - Visione Originalità}
\textbf{Attori principali}
\begin{itemize}
    \item Utente
\end{itemize}
\textbf{Attori secondari}
\begin{itemize}
    \item LLM
\end{itemize}
\textbf{Precondizioni}
\begin{itemize}
    \item L'utente è connesso al sistema
\end{itemize}
\textbf{Postcondizioni}
\begin{itemize}
    \item VERIFICA o RISCRIVE chiedere con MAIL
\end{itemize}
\textbf{Scenario principale}
\begin{enumerate}
    \item L'utente attiva il comando che riguardo il "Cappello Verde"
    \item ...
\end{enumerate}

\subsubsection{UC8.5 - Visione Logica e Struttura}
\textbf{Attori principali}
\begin{itemize}
    \item Utente
\end{itemize}
\textbf{Attori secondari}
\begin{itemize}
    \item LLM
\end{itemize}
\textbf{Precondizioni}
\begin{itemize}
    \item L'utente è connesso al sistema
\end{itemize}
\textbf{Postcondizioni}
\begin{itemize}
    \item VERIFICA o RISCRIVE chiedere con MAIL
\end{itemize}
\textbf{Scenario principale}
\begin{enumerate}
    \item L'utente attiva il comando che riguardo il "Cappello Blu"
    \item ...
\end{enumerate}

\subsection{UC9 - Eliminazione nota}
\begin{center}
\begin{tikzpicture} 
\begin{umlsystem}[x=6, fill=white!10]{Sistema}
  \umlusecase[x=0, y=4, name=eliminazione]{UC9 - Eliminazione nota}
\end{umlsystem}

% Attori
\umlactor[x=0, y=4]{Utente}

% Associazioni Utente
\umlassoc{Utente}{eliminazione}
\end{tikzpicture}
\end{center}
\textbf{Attori principali}
\begin{itemize}
    \item Utente
\end{itemize}
\textbf{Precondizioni}
\begin{itemize}
    \item L'utente è connesso al sistema
    \item L'utente ha selezionato una nota
\end{itemize}
\textbf{Postcondizioni}
\begin{itemize}
    \item La nota selezionata viene eliminata dal filesystem dell'applicazione
\end{itemize}
\textbf{Scenario principale}
\begin{enumerate}
    \item L'utente seleziona la nota che vuole eliminare
    \item La nota viene eliminata dal filesystem dell'applicazione
\end{enumerate}


\newpage
\begin{center}
    
\begin{tikzpicture}
    \begin{umlsystem}[x=0,y=0]{Sistema}
    \tikzumlset{font=\small}
    
        \umlactor[x=-5,y=-1.5,scale=1.5]{Utente}
        \umlactor[x=3.5,y=-1.5,scale=0.75]{LLM}
        \umlactor[x=4.5,y=-3,scale=0.75]{Sistema Persistenza}
        
        \umlusecase[name=UC1,x=0,y=0,align=center]{Editor \\ Testo MarkDown}
        \umlusecase[name=UC2,x=0,y=-1.5,align=center]{Funzioni AI}
        \umlusecase[name=UC3,x=0,y=-3,align=center]{Persistenza Note}
        
        \umlassoc{Utente}{UC1}
        \umlassoc{Utente}{UC2}
        \umlassoc{Utente}{UC3}
        \umlassoc{LLM}{UC2}
        \umlassoc{Sistema Persistenza}{UC3}

    \end{umlsystem}
\end{tikzpicture}

\end{center}

\vspace{10mm}
\begin{center}

\begin{tikzpicture}
    \begin{umlsystem}[x=0,y=0]{Editor Testo MarkDown}
    \tikzumlset{font=\small}
    
        \umlactor[x=-5,y=-1.5,scale=1.5]{Utente}
        
        \umlusecase[name=UC11,x=0,y=0,align=center]{Editing Testo}
        \umlusecase[name=UC12,x=0,y=-1.5,align=center]{Formattazione Testo} 
        \umlusecase[name=UC13,x=0,y=-3,align=center]{Gestione Layout}
        
        \umlassoc{Utente}{UC11}
        \umlassoc{Utente}{UC12}
        \umlassoc{Utente}{UC13}
    
    \end{umlsystem}
\end{tikzpicture}

\end{center}

\vspace{10mm}

\begin{center}

\begin{tikzpicture}
    \begin{umlsystem}[x=0,y=0]{Funzioni AI}
    \tikzumlset{font=\small}
    
        \umlactor[x=-5,y=-3,scale=1.5]{Utente}
        \umlactor[x=5,y=-3,scale=1]{LLM}

        \umlusecase[name=UC21,x=0,y=0,align=center]{Generazione Testo \\ da Prompt}
        \umlusecase[name=UC22,x=0,y=-1.5,align=center]{Riassunto \\ Testo}
        \umlusecase[name=UC23,x=0,y=-3,align=center]{Riscrittura \\ Testo} 
        \umlusecase[name=UC24,x=0,y=-4.5,align=center]{Traduzione \\ Testo}
        \umlusecase[name=UC25,x=0,y=-6,align=center]{Critica \\ Testo}
        
        \umlassoc{Utente}{UC21}
        \umlassoc{Utente}{UC22}
        \umlassoc{Utente}{UC23}
        \umlassoc{Utente}{UC24}
        \umlassoc{Utente}{UC25}

        \draw [tikzuml association style] (LLM) -- (2.6,-3);
    
    \end{umlsystem}
\end{tikzpicture}

\end{center}

\vspace{10mm}

\begin{center}

\begin{tikzpicture}
    \begin{umlsystem}[x=0,y=0]{Persistenza Note}
    \tikzumlset{font=\small}
    
        \umlactor[x=-5,y=-1.5,scale=1.5]{Utente}
        \umlactor[x=4.5,y=-1.5,scale=0.75]{Sistema Persistenza}
        
        \umlusecase[name=UC31,x=0,y=0,align=center]{Salvataggio Nota}
        \umlusecase[name=UC32,x=0,y=-1.5,align=center]{Apertura Nota} 
        \umlusecase[name=UC33,x=0,y=-3,align=center]{Eliminazione Nota}
        
        \umlassoc{Utente}{UC31}
        \umlassoc{Utente}{UC32}
        \umlassoc{Utente}{UC33}

        \draw [tikzuml association style] (Sistema Persistenza) -- (2.65,-1.5);
    
    \end{umlsystem}
\end{tikzpicture}

\end{center}

\vspace{10mm}

\begin{center}

\begin{tikzpicture}
    \begin{umlsystem}[x=0,y=0]{Riscrittura Testo}
    \tikzumlset{font=\small}
    
        \umlactor[x=-5,y=-2.3,scale=1.5]{Utente}
        \umlactor[x=5,y=-2.3,scale=1]{LLM}

        \umlusecase[name=UC231,x=0,y=0,align=center]{Miglioramento \\ Lessico}
        \umlusecase[name=UC232,x=0,y=-1.5,align=center]{Correzione \\ Grammatica}
        \umlusecase[name=UC233,x=0,y=-3,align=center]{Variazione \\ Stile Espositivo} 
        \umlusecase[name=UC234,x=0,y=-4.5,align=center]{Modifica \\ Estensione}
        
        \umlassoc{Utente}{UC231}
        \umlassoc{Utente}{UC232}
        \umlassoc{Utente}{UC233}
        \umlassoc{Utente}{UC234}

        \draw [tikzuml association style] (LLM) -- (2.4,-2.3);
    
    \end{umlsystem}
\end{tikzpicture}

\end{center}

\vspace{10mm}

\begin{center}

\begin{tikzpicture}
    \begin{umlsystem}[x=0,y=0]{Critica Testo}
    \tikzumlset{font=\small}
        \umlactor[x=-7,y=-5,scale=1.5]{Utente}
        \umlactor[x=6,y=-5,scale=1]{LLM}
        
        \umlusecase[name=UC251,x=0,y=0,align=center]{\textbf{Visione Emotiva} \\ (Cappello Rosso)}
        \umlusecase[name=UC252,x=0,y=-2,align=center]{\textbf{Visione Neutra} \\ (Cappello Bianco)}
        \umlusecase[name=UC253,x=0,y=-4,align=center]{\textbf{Visione Ottimista} \\ (Cappello Giallo)}
        \umlusecase[name=UC254,x=0,y=-6,align=center]{\textbf{Visione \textit{Devil's Advocate}} \\ (Cappello Nero)}
        \umlusecase[name=UC255,x=0,y=-8,align=center]{\textbf{Visione Originalità} \\ (Cappello Verde)}
        \umlusecase[name=UC256,x=0,y=-10,align=center]{\textbf{Visione Logica e Strutturata} \\ (Cappello Blu)}
        
        \umlassoc{Utente}{UC251}
        \umlassoc{Utente}{UC252}
        \umlassoc{Utente}{UC253}
        \umlassoc{Utente}{UC254}
        \umlassoc{Utente}{UC255}
        \umlassoc{Utente}{UC256}
        \draw [tikzuml association style] (LLM) -- (4.1,-5);
        
    \end{umlsystem}
\end{tikzpicture}

\end{center}







\end{document}