\documentclass[a4paper,10pt]{article}

\usepackage[hidelinks]{hyperref}
\usepackage{float}

% Lingua 
\usepackage[utf8]{inputenc}
\usepackage[italian]{babel}

% Margini
\usepackage{geometry}
\geometry{a4paper, top=3cm, bottom=3cm, left=3cm, right=3cm}
\setlength{\parindent}{0pt}

\usepackage{graphicx}   % per immagini
\usepackage{hyperref}   % per link
\usepackage{caption}    % per didascalie
\captionsetup[table]{labelformat=empty}
\usepackage{ifthen}     % per \ifthenelse


\usepackage{array}      % per tabelle
\newcolumntype{M}[1]{>{\centering\arraybackslash}m{#1}}

\usepackage{datetime}

% Definisci un formato personalizzato
\newdateformat{italianformat}{\THEDAY~\monthname[\THEMONTH]~\THEYEAR}
\newdate{dataverbale}{20}{11}{2025}

% Definizione del formato con le barre
\newdateformat{slashdate}{\twodigit{\THEDAY}/\twodigit{\THEMONTH}/\THEYEAR}

% Per logo in ogni pagina
\usepackage{eso-pic}
\usepackage{tikz}
\AddToShipoutPictureBG{%
  \ifthenelse{\value{page}>1}{%
    \begin{tikzpicture}[remember picture, overlay]
      % Icona
      \node[anchor=north west, inner sep=0pt] at ([xshift=28pt,yshift=-28pt]current page.north west) 
        {\includegraphics[width=1cm, height=1cm]{resources/Tool-8_icon_white.png}};
      
      % Riga
      \draw[black, line width=0.5pt] ([xshift=28pt + 1cm, yshift=-28pt - 0.6cm]current page.north west) 
            -- ++(1.8cm, 0);
      % Testo
      \node[anchor=west, text=black, font=\small] at ([xshift=28pt + 1cm, yshift=-28pt - 0.4cm]current page.north west) 
        {Glossario};
    \end{tikzpicture}
  }{}
}


%------Ridefinizione maketitle-------
\makeatletter
\renewcommand{\maketitle}{%
  \vspace*{\fill}
  \begin{center}
    {\LARGE \bfseries \@title \par}
    \vskip 1em
    {\large \@author \par}
    \vskip 1em
    {\large\slashdate{\displaydate{dataverbale}} \par}
  \end{center}
  \vspace*{\fill}
}
\makeatother
%-----------------------------------

\title{%
  \includegraphics[width=10cm]{resources/Tool-8_white_2.png} \\[1.5ex]
  \textbf{\LARGE Glossario}
}
\author{}
\date{\dataverbale}
\begin{document}
\tikz[remember picture,overlay] \node[opacity=0.2,inner sep=0pt] at (current page.center){\includegraphics[width=\paperwidth,height=\paperheight]{resources/Abstract_lines.png}};

\maketitle

\newpage

\section*{Revisioni}
\begin{table}[h!]
\centering
\begin{tabular}{|M{0.8cm}|p{2.2cm}|p{5.7cm}|M{1.8cm}|p{2.2cm}|}
  \hline
  \textbf{Ver.} & \textbf{Autore} & \textbf{Modifiche} & \textbf{Data} & \textbf{Verifica} \\
  
  \hline
  0.1 & Stefano Maso & Prima versione, struttura del documento e primi termini & 20/11/2025 &  \\
  \hline
\end{tabular}
\label{tab:revisioni}
\end{table}

\newpage

\tableofcontents

\newpage
\section{A}
\subsection{Analisi dei Requisiti}
Si tratta del processo attraverso il quale vengono individuate le esigenze e le funzionalità di un sistema software. L'obiettivo dell'analisi dei requisiti è quella di capire che cosa il sistema deve fare (requisiti funzionali) e con quali caratteristiche deve farlo (requisiti non funzionali), garantendo chiarezza, coerenza e verificabilità.
\newpage
\section{B}
\newpage
\section{C}
\subsection{Candidatura}
Indica l'atto attraverso il quale uno studente o un gruppo di studenti mostra formalmente il proprio interesse alla partecipazione e allo sviluppo di un progetto proposto da un'azienda.
\subsection{Capitolato}
È un documento ufficiale che descrive in modo dettagliato le caratteristiche, le condizioni e i requisiti di un progetto fungendo da riferimento durante lo sviluppo del progetto.
\subsection{Casi d'uso}
Si tratta di una descrizione strutturata che descrive come gli attori (utenti) interagiscano con il sistema, andando a rappresentare sequenze di azioni che mostrano come debba essere utilizzato il sistema.
\newpage
\section{D}
\subsection{Diagramma dei casi d'uso}
Rappresentazione grafica delle interazioni tra attori e il sistema software. Serve per definire i requisiti funzionali, chiarire cosa il sistema deve fare e fornire una base per lo sviluppo.
\newpage
\section{E}
\newpage
\section{F}
\newpage
\section{G}
\newpage
\section{H}
\newpage
\section{I}
\newpage
\section{J}
\newpage
\section{K}
\newpage
\section{L}
\newpage
\section{M}
\newpage
\section{N}
\newpage
\section{O}
\newpage
\section{P}
\newpage
\section{Q}
\newpage
\section{R}
\newpage
\section{S}
\newpage
\section{T}
\newpage
\section{U}
\newpage
\section{V}
\subsection{Verbale}
Documento che riporta le discussioni, le decisioni e gli eventi avvenuti durante una riunione o un incontro. Permette ai partecipanti di ricordare e consultare ciò che è avvenuto. Può comprendere una lista degli argomenti trattati, i partecipanti, le decisioni prese, le azioni future da svolgere ed eventuali commenti aggiuntivi.
\newpage
\section{W}
\subsection{Way of Working}
Si tratta di un insieme di processi, regole, pratiche e strumenti che un team utilizza per svolgere il proprio lavoro in modo efficace e coordinato.
\newpage
\section{X}
\newpage
\section{Y}
\newpage
\section{Z}
\end{document}